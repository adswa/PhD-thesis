% !TeX root = ../main-english.tex
% !TeX spellcheck = en-US
% !TeX encoding = utf8
% -*- coding:utf-8 mod:LaTeX -*-

%This smart spell only works if no changes have been made to the chapter
%using the options proposed in preambel/chapterheads.tex.
\setchapterpreamble[u]{%
	\dictum[John Claerbout, paraphrased by Buckheit \& Donoho]{An article about computational science in a scientific publication is not the scholarship itself, it is merely advertising of the scholarship. The actual scholarship is the complete software
	development environment and the complete set of instructions which generated the figures.}
}


\chapter{Ensuring computational reproducibility across computational environments}
\label{chap:k3}


Partially fueled by funding mandates (CITATION needed) and research curricula founded within the Open Science Movement (CITATION NEEDED), practices of publishing reproducibly and reproducing published results are becoming more frequent.
Grass-roots movements such as Reprohack (\href{https://www.reprohack.org/}{www.reprohack.org}) or the ``Ten Years Reproducibility Challenge'' (\href{https://rescience.github.io/ten-years/}{rescience.github.io/ten-years}) train researchers to check published studies for reproducibility.
Widespread sharing of code and data allows researchers to verify, reuse, and improve upon past work of others (CITATION NEEDED).
Consequently, though, attempts to reproduce previous studies often happen in different computational environments than those that originally created the results in question.
Ensuring computational reproducibility across computational environments is, however, a difficult technical challenge.
This following chapter outlines first its challenges, particularly in the field of neuroimaging, then its opportunities, and lastly an implementation to ensure computational reproducibility across computational environments.
Parts of this chapter were published as \citet{wagner2022fairly}: ``FAIRly big: A framework for computationally reproducible processing of large-scale data'' and are appropriately marked as such.



\pagebreak

\section{FAIRly big: A framework for computationally reproducible processing of large-scale data}
\pagebreak
% !TeX root = ../main-english.tex
% !TeX spellcheck = en-US
% !TeX encoding = utf8
% -*- coding:utf-8 mod:LaTeX -*-

%This smart spell only works if no changes have been made to the chapter
%using the options proposed in preambel/chapterheads.tex.
\setchapterpreamble[u]{%
	\dictum[Isaac Newton]{If I have seen further it is by standing on the shoulders of Giants}
}

\chapter{Discussion}
\label{discussion}

Look mom, some text!



File content transport across this network is possible via versatile transport logistics that allow for local or remote data hosting. This can enable data transports on systems with too little available disk space for multiple copies, allow redundant storage to be configured, interoperate with hosting services to publish results, or reconfigure data access when remote hosting locations change—without needing to alter the data representation in the dataset.

With these technical features, how and where data are stored (e.g., local, encrypted storage; remote, cloud-based hosting) becomes orthogonal to how and where computations are performed (e.g., on-site compute cluster; remote cloud-computing service). This allows our framework to bootstrap ephemeral (short-lived) workspaces for individual computational jobs, retrieve only relevant processing elements (e.g., subsets of input data), and extend the DataLad datasets’ revision history with their results and process provenance (Figure 1c). This, in turn, opens the possibility for parallel and version controlled analysis progression, using a distributed network of temporary clones. Results and revision histories can be merged to form a full processing history, in a similar way to how code is collaboratively developed with distributed version control tools




The modular assembly enables (re)use of independently maintained components, separates access modalities such that access-restricted input data does not impair sharing of less sensitive outcomes, and guarantees precise identification of processing components, regardless of whether a particular dataset consumer has access to a given component.




\section{Conclusion}

\section{Limitations}


\section{Opportunities}


